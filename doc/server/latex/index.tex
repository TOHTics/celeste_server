\hypertarget{index_Description}{}\section{Description}\label{index_Description}
This project consists in the monitoring of \char`\"{}samples\char`\"{} sent by some {\ttfamily Logger} device that measures various different types of these samples. In the Sun\+Spec Data Exchange (S\+DX) terminology, these samples are called {\ttfamily Points}. In general terms, the {\ttfamily Logger} sends measurements encoded in X\+ML to the server, where the server is allowed to do anything that it wants with the data; such as inserting it into a database.\hypertarget{index_Points}{}\section{Points}\label{index_Points}
Points are measurements taken by {\ttfamily Models}. These points can be things such as\+:
\begin{DoxyItemize}
\item Amperage
\item Wattage
\item Latitude
\item Longitude
\item Altitude
\item Temperature The exact interpretation and encoding of these points is to be defined in the folder {\ttfamily models}.
\end{DoxyItemize}

Various points can be taken by one model. This means that for every model {\ttfamily M}, there are {\ttfamily N} points {\ttfamily P} that are sampled by it. In a diagram\+: 
\begin{DoxyCode}
  is sampled by
P @-------------\(\backslash\)
P @--------------\(\backslash\)
.                 \(\backslash\)
.                 [M]
.                 /
P @--------------/
P @-------------/
\end{DoxyCode}
 \hypertarget{index_Models}{}\section{Models}\label{index_Models}
Models are objects such as meters, temperatures sensors, G\+PS, etc. They are the ones in charge of sampling the {\ttfamily Points}. Please refer to the {\ttfamily Points} section to understand what a point is. It is clear that a G\+PS cannot measure things such as Amperage or Temperature, but it can measure Latitude, Longitude and Altitude. This is important to note as it is necessary to define exactly what the models can measure. These models are \char`\"{}contained\char`\"{} or \char`\"{}read by\char`\"{} {\ttfamily Devices}. Many models can be read by one device. That is, a device {\ttfamily D} can read {\ttfamily N} models {\ttfamily M} at once. In a diagram this looks like\+:


\begin{DoxyCode}
    is read by
M @-------------\(\backslash\)
M @--------------\(\backslash\)
.                 \(\backslash\)
.                 [D]
.                 /
M @--------------/
M @-------------/
\end{DoxyCode}
 The models \char`\"{}contain\char`\"{} many points.\hypertarget{index_Devices}{}\section{Devices}\label{index_Devices}
Devices are to be thought of as \char`\"{}black boxes\char`\"{} that somehow sample measurements using the {\ttfamily Models}. Their sole function is to sample the measurements and make sure those samples reach the server. Think of devices as black boxes that contain {\ttfamily N} models, therefore encapsulating the electronics behind it and their output is an X\+ML document. In a diagram this might look like\+:


\begin{DoxyCode}
@-----------@
| M . . . M |            @ o o o o o @
| . .     . |    XML     8           8
| .   .   . | ---------> 8  Server   8
| .     . . |  Internet  8           8
| M . . . M |            8 o o o o o @
@-----------@
\end{DoxyCode}


We note that we only think these \char`\"{}`\+Devices`\char`\"{} as black boxes containing models. However, it might not (and it will be most likely) that this is not the case physically; the {\ttfamily Models} could be scattered around and there are cables connected to the {\ttfamily Device}. 